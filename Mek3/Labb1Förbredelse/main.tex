\documentclass[a4paper]{article}
\usepackage[utf8]{inputenc}

\usepackage[swedish]{babel}
\usepackage[pdftex]{graphicx}
\usepackage{amsmath}
\usepackage{float}
\usepackage{caption}
\usepackage{subcaption}
\usepackage{xcolor}
\usepackage[
backend=biber,
style=apa
]{biblatex}

\addbibresource{referenser.bib}

\newcommand{\ihat}{\boldsymbol{\hat{\textbf{\i}}}}
\newcommand{\jhat}{\boldsymbol{\hat{\textbf{\j}}}}
\newcommand{\khat}{\boldsymbol{\hat{\textbf{k}}}}
\newcommand{\placeholder}{{\huge\textbf{\textcolor{red}{Remember to put something good here!!!}}}}

\textwidth 155mm \oddsidemargin -0mm
\parskip 5mm
\parindent 0mm

\title{Förberedelser labb 1}
\author{Anton Lindbro}
\date{\today}

\begin{document}
\maketitle

\section{Lagrangian}

\begin{figure}[H]
    \begin{small}
        \begin{center}
            \includegraphics[width=0.95\textwidth]{figures/problemuppställning.jpg}
        \end{center}
        \caption{}
        \label{fig:uppställning}
    \end{small}
\end{figure}


För lagrangianen behöver vi T och U. Från fig \ref{fig:uppställning} kan vi plocka fram T och U.
T kommer från pendlarnas kinetiska energi som kommer från deras rotationella energi. 

\begin{align}
    T = \frac{1}{2}I_O\omega_1^2 + \frac{1}{2}I_O\omega_2^2
\end{align}

Där m är pendelns massa, $I_O$ är pendelns tröghetsmoment runt punkten O och $\omega_i$ är pendlarnas respektive hastigheter och vinkelhastigheter. Generaliserade koordinater väljs till $\phi, \theta$ och vi kan då uttrycka nödvändiga storheter i dessa koordinater.

\begin{align}
    \omega_1 = \dot{\theta} \\
    \omega_2 = \dot{\phi}
\end{align}

Det ger 

\begin{align}
    T =  \frac{1}{2}I_O\dot{\theta}^2 + \frac{1}{2}I_O\dot{\phi}^2
\end{align}

U kommer nu dels från gravitations potentialen och dels från fjäderns potential.

\begin{align}
    U = mgh_1 + mgh_2 + \frac{1}{2}kd^2
\end{align}

Där $h_i$ är höjden för respektive masscentrum och d är längd förändringen av fjädern.

\begin{align}
    h_1 = L(1-\cos\theta)\\
    h_2 = L(1-\cos\phi)\\
    d = l(\sin\theta-\sin\phi)
\end{align}

Det ger

\begin{align}
    U = mgL(1-\cos\theta) + mgL(1-\cos\phi) + \frac{1}{2}kl^2(\sin\theta-\sin\phi)^2
\end{align}

Vid små svängningar använder vi ledande ordning 2. Därmed blir lagraniganen till ledande ordning

\begin{align}
    L \approx \frac{1}{2}I_O\dot{\theta}^2 + \frac{1}{2}I_O\dot{\phi}^2 - mgL\left (\frac{\theta^2}{2}+\frac{\phi^2}{2}\right ) - \frac{1}{2}kl^2(\theta^2 +\phi^2-2\theta\phi)
\end{align}

\section{Rörelse ekvationer}

Från detta kan vi få rörelse ekvationerna
\begin{align}
    \frac{\partial L}{\partial \theta} &= -mgL\theta - kl^2\theta + kl^2\phi \\
    \frac{\partial L}{\partial \dot{\theta}} &= I_O\dot{\theta} \\
    \frac{\partial L}{\partial \phi} &= -mgL\phi - kl^2\phi + kl^2\theta \\
    \frac{\partial L}{\partial \dot{\phi}} &= I_O \dot{\phi}
\end{align}

\begin{align}
    \frac{d}{dt}(\frac{\partial L}{\partial \dot{\theta}}) - \frac{\partial L}{\partial \theta} = I_O\ddot{\theta} + mgL\theta +kl^2\theta - kl^2\phi = 0\\
    \frac{d}{dt}(\frac{\partial L}{\partial \dot{\phi}}) - \frac{\partial L}{\partial \phi} = I_O\ddot{\phi} + mgL\phi + kl^2\phi - kl^2\theta = 0
\end{align}

Samlar vi termer och skriver om

\begin{align}
    \ddot{\theta}I_O + \theta(mgL+kl^2) - kl^2\phi = 0 \\
    \ddot{\phi}I_O + \phi(mgL+kl^2) - kl^2\theta = 0
\end{align}

\section{Lösning}

Vi förväntar oss en svängnings rörelse så vi ansätter det med $\theta = B_1e^{i\omega t}$ och $\phi = B_2e^{i\omega t}$. Sätter vi in det i ekv. 17 och 18 fås

\begin{align}
    -\omega^2 B_1e^{i\omega t} I_O + B_1e^{i\omega t}(mgL+kl^2) - kl^2 B_2e^{i\omega t} = 0 \\
    -\omega^2 B_2e^{i\omega t} I_O + B_2e^{i\omega t} (mgL+kl^2) - kl^2 B_1e^{i\omega t} = 0
\end{align}

För att vi ska få en svängningsrörelse måste vi anta $e^{i \omega t} \neq 0$, samlar vi termer får vi då ekvationerna

\begin{align}
    B_1(-\omega^2I_O + (mgL+kl^2)) - B_2kl^2 = 0 \\
    B_2(-\omega^2I_O + (mgL+kl^2)) - B_1kl^2 = 0
\end{align}

Detta kan vi skriva om till en matris ekvation

\begin{align}
    \begin{pmatrix}
        -\omega^2I_O + (mgL+kl^2) & -kl^2 \\
        -kl^2 & -\omega^2I_O + (mgL+kl^2)
    \end{pmatrix} \begin{pmatrix}
        B_1 \\
        B_2
    \end{pmatrix} = \mathbf{0}
\end{align}

För att denna ska ha icke triviala lösningar måste determinanten av koefficient matrisen vara 0.

\begin{align}
    \begin{vmatrix}
        -\omega^2I_O + (mL+kl^2) & -kl^2 \\
        -kl^2 & -\omega^2 I_O + (mgL+kl^2)
    \end{vmatrix} = \mathbf{0}
\end{align}

Detta ger ekvationerna

\begin{align}
    (-\omega^2I_O + (mgL+kl^2))^2 - (-kl^2)^2 = 0 \\
    (-\omega^2I_O + (mgL+kl^2))^2 = (kl^2)^2\\
    -\omega^2I_O + (mgL+kl^2) = \pm kl^2 \\
    \omega^2 = \frac{\mp kl^2 + mgL + kl^2}{I_O}
\end{align}

Detta ger egenfrekvenserna

\begin{align}
    \omega_- = \sqrt{\frac{mgL}{I_O}} \\
    \omega_+ = \sqrt{\frac{mgL + 2kl^2}{I_O}}
\end{align}

Nu när vi har egenfrekvenserna behöver vi ta fram egenmoderna. Adderar vi 17 och 18 får vi

\begin{align}
    \ddot{\theta}I_O + \theta(mgL+kl^2) - kl^2 \phi + I_O\ddot{\phi} + \phi (mgL+kl^2) - kl^2 \theta = 0 \\
    I_0(\ddot{\theta} + \ddot{\phi}) + (mgL+kl^2)(\theta+ \phi) - kl^2(\theta + \phi) = 0 \Rightarrow q_+ = (\theta + \phi)
\end{align}

Subtraherar vi dom fås

\begin{align}
    \ddot{\theta}I_O + \theta(mgL+kl^2) - kl^2 \phi - I_O\ddot{\phi} - \phi (mgL+kl^2) + kl^2 \theta \\
    I_0(\ddot{\theta}-\ddot{\phi}) + (mgL+kl^2)(\theta - \phi) - kl^2(\theta - \phi) \Rightarrow q_ = (\theta-\phi)
\end{align}

Egenmoderna svänger med egenfrekvenserna.

\begin{align}
    q_+(t) = A\cos(\omega_+ t) + B\sin(\omega_+ t) \\
    q_-(t) = C\cos(\omega_- t) + D\sin(\omega_- t)
\end{align}

$\theta$ och $\phi$ ges av $q_+$ och $q_-$ enligt

\begin{align}
    \theta = \frac{1}{2}(q_+ + q_-)\\
    \phi = \frac{1}{2}(q_+ - q_-)
\end{align}

Det ger den allmänna lösningen för $\theta$ och $\phi$

\begin{align}
    \theta(t) = \frac{1}{2}(A\cos(\omega_+ t) + B\sin(\omega_+ t) + C\cos(\omega_- t) + D\sin(\omega_- t))\\
    \phi(t) = \frac{1}{2}(A\cos(\omega_+ t) + B\sin(\omega_+ t) - C\cos(\omega_- t) - D\sin(\omega_- t))
\end{align}

Vilket ger 4 konstanter vilket är rimligt då vi har två andragrads diff ekvationer. 

\section{Specialfall}

\subsection*{A}

Begynnelse värden
\begin{align}
    \theta(0) = \alpha\\
    \dot{\theta}(0) = 0 \\
    \phi(0) = \alpha \\
    \dot{\phi} = 0
\end{align}

Detta ger

\begin{align}
    q_+(0) = 2 \alpha \\
    q_-(0) = 0 \\
    \dot{q_+} = 0 \\
    \dot{q_-} = 0 \\
\end{align}

Detta ger

\begin{align}
    q_+(0) = A = 2 \alpha \\
    \dot{q_+}(0) = -B = 0 \\
    q_-(0) = C = 0 \\
    \dot{q_-} = -D = 0
\end{align}

Så vi får inget - bara + vilket var väntat

\subsection*{B}
Begynnelse värden
\begin{align}
    \theta(0) = \alpha\\
    \dot{\theta}(0) = 0 \\
    \phi(0) = -\alpha \\
    \dot{\phi} = 0
\end{align}

Detta ger

\begin{align}
    q_+(0) =  \\
    q_-(0) = 2 \alpha \\
    \dot{q_+} = 0 \\
    \dot{q_-} = 0 \\
\end{align}

Detta ger

\begin{align}
    q_+(0) = A = 0\\
    \dot{q_+}(0) = -B = 0 \\
    q_-(0) = C = 2 \alpha \\
    \dot{q_-} = -D = 0
\end{align}

Nu får vi inget + bara - vilket också var väntat

\subsection*{C}

Begynnelse värden
\begin{align}
    \theta(0) = \alpha\\
    \dot{\theta}(0) = 0 \\
    \phi(0) = 0 \\
    \dot{\phi} = 0
\end{align}

detta ger

\begin{align}
    q_+(0) = \alpha \\
    q_-(0) = \alpha \\
    \dot{q_+} = 0 \\
    \dot{q_-} = 0 \\
\end{align}

\begin{align}
    q_+(0) = A = \alpha\\
    \dot{q_+}(0) = -B = 0 \\
    q_-(0) = C = \alpha \\
    \dot{q_-} = -D = 0
\end{align}

Detta ger konstant blanding av båda med samma amplitud vilket också var väntat.

\end{document}

