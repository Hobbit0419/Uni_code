\documentclass[a4paper]{article}
\usepackage[utf8]{inputenc}

\usepackage[swedish]{babel}
\usepackage[pdftex]{graphicx}
\usepackage{amsmath}
\usepackage{float}
\usepackage{caption}
\usepackage{subcaption}
\usepackage{xcolor}
\usepackage[
backend=biber,
style=apa
]{biblatex}

\addbibresource{referenser.bib}

\newcommand{\ihat}{\boldsymbol{\hat{\textbf{\i}}}}
\newcommand{\jhat}{\boldsymbol{\hat{\textbf{\j}}}}
\newcommand{\khat}{\boldsymbol{\hat{\textbf{k}}}}
\newcommand{\placeholder}{{\huge\textbf{\textcolor{red}{Remember to put something good here!!!}}}}

\textwidth 155mm \oddsidemargin -0mm
\parskip 5mm
\parindent 0mm

\title{Inlämmning 1 mekanik III}
\author{Anton Lindbro}
\date{\today}

\begin{document}

\maketitle

\section*{Uppgift 1}

\subsection*{Uppställning}

\begin{figure}[H]
    \begin{small}
        \begin{center}
            \includegraphics[width=0.8\textwidth]{figures/Screenshot 2025-02-01 at 11-45-11 mek3 - mek3.pdf.png}
        \end{center}
        \caption{Problem uppställning}
        \label{fig:uppställning}
    \end{small}
\end{figure}

Problemet består av ett snöre med en fin gammal tekopp i ena änden och ett gem i den andra. Snörret löper friktionsfritt över punkten A. Vid $t=0$ släpps gemet och uppgiften är att hitta rörelsen för gemet som funktion av tiden.

För att kunna göra en modell av gemets rörelse så behöver vi göra några idealiseringar.

\begin{itemize}
    \item Icke elastiskt snöre med längden $l$
    \item Snöret löper friktionsfritt över A
    \item Diametern av A är mycket liten $\Rightarrow r+z=l$ gäller alltid
    \item Inget luftmotstånd
\end{itemize}

Jag har valt att använda polära koordinater för att beskriva systemet. Detta då vi antar att koppen faller vertikalt neråt. Jag har valt att sätta r som avståndet mellan gemet och punkten A, och $\theta$ vara vinkeln mellan snöret som sitter fast i gemet och ursprungsläget som visas i figuren. 

Dom sambanden som används är 

\begin{align}
    z+r=l\\
    T = \frac{1}{2}mv^2\\
    U = -mgh\\
    L = T-U
\end{align}

\subsection*{Plan}

\begin{enumerate}
    \item Hitta kinetiskt och potentiell energi
    \item Skriv upp lagrangianen
    \item Beräkna relevanta derivator
    \item Skriv upp rörelse ekvationerna
    \item Lös dessa på lämpligt sätt
    \item Plotta lösningen
\end{enumerate}

\subsection*{Utförande}

Den kinetiska energin för systemet är summan av den kinetiska energin för dom båda kropparna.

\begin{equation}
    T = \frac{1}{2} Mv_1^2 + \frac{1}{2}mv_2^2
\end{equation}

Utifrån figuren kan vi ta fram $v_1 och v_2$

\begin{align}
    v_1 = \dot{z} = -\dot{r} \\
    v_2 = \sqrt{\dot{r}^2 + r^2\dot{\theta}^2}
\end{align}

Sätter vi in det i ekv. 5 och förenklar får vi.

\begin{equation}
    T = \frac{1}{2}(\dot{r}^2(M+m)+ mr^2\dot{\theta}^2)
\end{equation}

För att ta fram den potentiella energin för de båda kropparna behöver vi ha det vertikala avståndet från noll nivån. För koppen blir det

\begin{equation}
    h = z = l - r
\end{equation}

För gemet behöver vi ha koll på vinkeln också

\begin{equation}
    h = -r\sin\theta
\end{equation}

Använder vi nu dessa höjderna får vi

\begin{equation}
    U = -Mg(l-r) - mgr\sin\theta
\end{equation}

Detta tillsammans med ekv.8 ger lagrangianen

\begin{equation}
    L = \frac{1}{2}(\dot{r}^2(M+m)+ mr^2\dot{\theta}^2) +Mg(l-r) + mgr\sin\theta
\end{equation}

Nu behöver vi beräkna lite derivator

\begin{align}
    \frac{\partial L}{\partial \theta} = -mgr\cos\theta \\
    \frac{\partial L}{\partial \dot{\theta}} = mr^2\dot{\theta}\\
    \frac{\partial L}{\partial r} = mr\dot{\theta}^2 - Mg + mg\sin\theta \\
    \frac{\partial L}{\partial \dot{r}} = \dot{r}(M+m)
\end{align}

Dessa ger os rörelse ekvationerna

\begin{align}
    \frac{d}{dt}(mr^2\dot{\theta}) + mgr\cos\theta = 0 \\
    \frac{d}{dt}(\dot{r}(M+m)) - mr\dot{\theta}^2 + Mg - mg\sin\theta = 0
\end{align}

Beräknar vi tids derivatorna får vi 

\begin{align}
    \ddot{\theta}r^2m + 2\dot{\theta}r\dot{r} + \sin\theta mgr = 0\\
    \ddot{r}(M+m)-mr\dot{\theta}^2 + g(M-m) = 0
\end{align}

För att få något hum om hurvida dessa rörelsekvationerna stämmer eller inte så gör vi en enhets analys

\begin{align}
    s^{-2}m^2kg + s^{-1}m m s^{-1} + kg m s^{-2} m = \frac{kg m^2}{s^2}\\
    ms^{-2} kg - kg m s^{-2} + ms^{-2} kg = \frac{kgm}{s^2} 
\end{align}

Så enheterna är konsekventa i dom två ekvationerna. Eftersom det är kopplade ickelinjära ekvationer använder jag mathematica för att lösa dom. Det ger en lösning som ser ut som följnade. 

Kommentar: Min numeriska lösning ger att $\theta$ är konstant så antingen är mina rörelse ekvationer fel eller så har jag skrivit nått fel i mathematica. Kan inte själv hitta några fel dock. 


\section*{Uppgift 2}

\subsection*{Uppställning}

\begin{figure}[H]
    \begin{small}
        \begin{center}
            \includegraphics[width=0.6\textwidth]{figures/Screenshot 2025-02-01 at 13-19-53 mek3 - mek3.pdf.png}
        \end{center}
        \caption{Problem uppställning}
        \label{fig:uppställning}
    \end{small}
\end{figure}

I denna uppgiften har vi en sattelit i omloppsbana kring jorden. Precis som alla satteliter utsätts den för luftmotstånd. Uppgiften är att plocka fram rörelse ekvationerna och lösa dessa under vissa förhållanden. 

Vi börjar med idealiseringarna
\begin{itemize}
    \item Luftmotståndet $F_l = -mk\mathbf{v}$ där k är en konstant
    \item k är liten $\Rightarrow$ långsam förändring i omloppsbanan
\end{itemize}

Jag väljer att återigen använda polära koordinater. Här använder vi istället sattelitens höjd över marken och vinkeln som syns i bilden. 

Sambanden som används

\begin{align}
    T = \frac{1}{2}mv^2\\
    U = -mgh\\
    \mathbf{F_l} = -mk\mathbf{v}\\
    L = T-U
\end{align}

\subsection*{Plan}

\begin{enumerate}
    \item Ta fram T och U för satteliten
    \item Ställ upp lagrangianen
    \item Beräkna derivator
    \item Ställ upp rörelsekvationerna med luftmotståndet som en generaliserad kraft
    \item Lös dom under idealiseringarna
\end{enumerate}

\subsection*{Utförande} 

Den kinetiska energin får vi som
\begin{equation}
    T = \frac{1}{2}m\mathbf{v}^2 = \frac{1}{2}m(\dot{r}^2+r^2\dot{\theta}^2)
\end{equation}

med r som avståndet från centrum och ut

Den potentiella energin är 

\begin{equation}
    U = mgr
\end{equation}

med noll för potentiell energi i origo

Detta ger lagrangianen

\begin{equation}
    L = \frac{1}{2}m(\dot{r}^2+r^2\dot{\theta}^2) - mgr
\end{equation}

Beräknar vi derivatorna

\begin{align}
    \frac{\partial L}{\partial \theta} = 0 \\
    \frac{\partial L}{\partial \dot{\theta}} = mr^2\dot{\theta} \\
    \frac{\partial L}{\partial r} =  mr\dot{\theta}^2 - mg\\
    \frac{\partial L}{\partial \dot{r}} = m\dot{r}\\
\end{align}

Det ger rörelsekvationerna

\begin{align}
    \frac{d}{dt}(\frac{\partial L}{\partial \dot{\theta}}) - \frac{\partial L}{\partial \theta} &= \frac{d}{dt}(mr^2\dot{\theta}) \\
    \frac{d}{dt}(\frac{\partial L}{\partial \dot{r}}) - \frac{\partial L}{\partial r} &= \frac{d}{dt}(m\dot{r}) - mr\dot{\theta}^2 + mg
\end{align}

För att få med luftmotståndet behöver vi beräkna de generaliserade krafterna för varje koordinat. Detta görs enligt följande

\begin{align}
    Q_{\theta} = -mk\mathbf{v} \cdot \frac{\partial \mathbf{v}}{\partial \dot{\theta}} = -mkr^2\dot{\theta} \\
    Q_r = -mk\mathbf{v} \cdot \frac{\partial \mathbf{v}}{\partial \dot{r}} = -mk\dot{r}
\end{align}



Beräknar vi tids derivatorna och sätter ekvationerna lika med respektive generaliserad kraft

\begin{align}
    mr^2\ddot{\theta} + 2mr\dot{r}\dot{\theta} = -mkr^2\dot{\theta}\\
    m\ddot{r} - mr\dot{\theta}^2 + mg = -mk\dot{r}
\end{align}

För en sanity check kollar vi enheterna i höger leden

\begin{align}
    kg m^2 s^{-2} + kg m m s^{-1} s^{-1} = \frac{kgm^2}{s^2} \\
    kg m s^{-2} - kg m s^{-2} kg m s^{-2} = \frac{kgm}{s^2}
\end{align}

Detta ger $[k] = s^{-1}$ i båda ekvationerna så enheterna stämmer och vi verkara vara på rätt spår. 


Kommentar:
Nu är jag lite osäker på hur jag ska gå vidare. Kan jag anta att $\dot{r} = 0$ och därmed få ett ODE system för $\theta$.

\end{document}

