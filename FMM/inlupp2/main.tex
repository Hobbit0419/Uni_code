\documentclass[a4paper]{article}
\usepackage[utf8]{inputenc}

\usepackage[swedish]{babel}
\usepackage[pdftex]{graphicx}
\usepackage{amsmath}
\usepackage{float}
\usepackage{caption}
\usepackage{subcaption}
\usepackage{xcolor}
\usepackage[
backend=biber,
style=apa
]{biblatex}

\addbibresource{referenser.bib}

\newcommand{\ihat}{\boldsymbol{\hat{\textbf{\i}}}}
\newcommand{\jhat}{\boldsymbol{\hat{\textbf{\j}}}}
\newcommand{\khat}{\boldsymbol{\hat{\textbf{k}}}}
\newcommand{\placeholder}{{\huge\textbf{\textcolor{red}{Remember to put something good here!!!}}}}

\textwidth 155mm \oddsidemargin -0mm
\parskip 5mm
\parindent 0mm

\title{Assignment 2 Mathematical methods in physics}
\author{Anton Lindbro}
\date{\today}

\begin{document}

\maketitle

\section{Problem 1}

\subsection{}

We are supposed to write a series solution for the given PDE. We have a string with fixed ends that is being driven by a timedependent force. The PDE is given by   
\begin{align*}
    \frac{\partial^2 u}{\partial t^2} = c^2 \frac{\partial^2 u}{\partial x^2} + f(x,t)
\end{align*}

Since we have fixed ends a sine series is appropriate for the solution. We can write that series as
\begin{align*}
    u(x,t) = \sum_{n=1}^{\infty} u_n(t) \sin \left ( \frac{n\pi x}{L} \right )
\end{align*}

\subsection{}

Now we need to compute the series coeffcients for a cosine series for the driving force. We can write the driving force as
\begin{align*}
    f(t) = \sum_{n=1}^{\infty} f_n \sin \left ( \frac{n\pi x}{L} \right )
\end{align*}

\begin{align*}
    f_n = \frac{2}{L} \int_{0}^{L} f(t) \sin \left ( \frac{n\pi x}{L} \right ) dx
\end{align*}

Since $f(t)$ is constant in x we get

\begin{align*}
    f_n &= \frac{2f(t)}{L} \int_{0}^{L} \sin \left ( \frac{n\pi x}{L} \right ) dx = \frac{2f(t)}{L} \left [ -\frac{L}{n\pi} \cos \left ( \frac{n\pi x}{L} \right ) \right ]_{0}^{L} = \frac{2f(t)}{n\pi} \left ( 1 - (-1)^n \right )\\
    f_n &= \frac{4f(t)}{n\pi} \text{ for n odd}
\end{align*}

\subsection{}

Now we substitute the series solutions into the PDE and solve for $u_n(t)$

\begin{align*}
    \sum_{n=1}^\infty \frac{\partial^2 u_n}{\partial t^2} \sin \left ( \frac{n\pi x}{L} \right ) = -\frac{c^2 \pi^2}{L^2} \sum_{n=1}^{\infty} u_n(t) n^2\sin \left ( \frac{n\pi x}{L} \right ) + \sum_{n=1}^{\infty}f_n \sin \left ( \frac{n\pi x}{L} \right )
\end{align*}

Now we cancel all $\sin$ terms and get

\begin{align*}
    \frac{\partial^2 u_n}{\partial t^2} = -\frac{c^2 \pi^2}{L^2} u_n(t) n^2 + f_n
\end{align*}

\subsection{}

If we set $f(t) = \sin{\omega t}$ we can solve the ODE for $u_n(t)$

\begin{align*}
    \frac{\partial^2 u_n}{\partial t^2} = -\frac{c^2 \pi^2}{L^2} u_n(t) n^2 + \frac{4\sin{\omega t}}{n\pi}
\end{align*}
\subsection{}

\subsection{}

\section{Problem 2}

\subsection{}

\subsection{}

\section{Problem 3}

\subsection{}

\subsection{}

\subsection{}

\subsection{}

\section{Problem 4}


\section{Problem 5}

\subsection{}

\subsection{}

\end{document}

