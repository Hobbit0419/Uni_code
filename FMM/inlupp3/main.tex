\documentclass[a4paper]{article}
\usepackage[utf8]{inputenc}

\usepackage[swedish]{babel}
\usepackage[pdftex]{graphicx}
\usepackage{amsmath}
\usepackage{float}
\usepackage{caption}
\usepackage{subcaption}
\usepackage{xcolor}
\usepackage[
backend=biber,
style=apa
]{biblatex}

\addbibresource{referenser.bib}

\newcommand{\ihat}{\boldsymbol{\hat{\textbf{\i}}}}
\newcommand{\jhat}{\boldsymbol{\hat{\textbf{\j}}}}
\newcommand{\khat}{\boldsymbol{\hat{\textbf{k}}}}
\newcommand{\placeholder}{{\huge\textbf{\textcolor{red}{Remember to put something good here!!!}}}}

\textwidth 155mm \oddsidemargin -0mm
\parskip 5mm
\parindent 0mm

\title{Homework 3 mathematical methods in physics}
\author{Anton Lindbro}
\date{\today}

\begin{document}

\maketitle

\section*{Problem 1}

Here we have a ball $B^3$ of radius $a$ , the boundary of this ball is a sphere $S^2$ diveded into a northern and southern hemisphere. The northern hemisphere is in contact with a heatbath at temperature T and thesouthern hemisphere is in contact with a heatbath at temperature 0. There are no heatsources inside the ball. At time t=0 the temperature is uniform and equal to $T/2$. We are asked to find the equilibrium temperature distribution in the ball. To find this we need to solve the following equation

\begin{align*}
    \nabla^2 u(\rho,\phi,\theta) = 0
\end{align*}

where $u$ is the temperature distribution, $\rho$ is the radius, $\theta$ is the polar angle and $\phi$ is the azimuthal angle. The boundary conditions are as follows:
\begin{align*}
    u(a,\theta,\phi) = T \text{ for } 0 \leq \theta < \pi/2 \\
    u(a,\theta,\phi) = 0 \text{ for } \pi/2 < \theta < \pi \\
\end{align*}

We can rewrite these using the Heaviside theta function $\theta_H(x)
$

\begin{align*}
    u(a,\theta,\phi) = T \theta_H(\pi/2 - \theta)
\end{align*}

We need to rewrite the Laplace equation in spherical coordinates. The Laplace operator in spherical coordinates is given by
\begin{align*}
    \nabla^2 u = \frac{1}{\rho^2} \frac{\partial}{\partial \rho} \left( \rho^2 \frac{\partial u}{\partial \rho} \right) + \frac{1}{\rho^2 \sin(\theta)} \frac{\partial}{\partial \theta} \left( \sin(\theta) \frac{\partial u}{\partial \theta} \right) + \frac{1}{\rho^2 \sin^2(\theta)} \frac{\partial^2 u}{\partial \phi^2}
\end{align*}

We can now separate variables $u(\rho,\theta,\phi) = r(\rho)t(\theta)f(\phi)$ and plug this into the Laplace equation aswell as dividing by $r(\rho)t(\theta)f(\phi)$. This gives us the following equation

\begin{align*}
    \frac{1}{r}\frac{1}{\rho^2} \frac{d}{d \rho} \left( \rho^2 \frac{d r(\rho)}{d \rho} \right) + \frac{1}{t}\frac{1}{\rho^2\sin(\theta)} \frac{d}{d \theta} \left( \sin(\theta) \frac{d t(\theta)}{d \theta} \right) + \frac{1}{f}\frac{1}{\rho^2 \sin^2(\theta)} \frac{d^2 f(\phi)}{d \phi^2} = 0
\end{align*}

We can now multiply by $\rho^2 \sin^2(\theta)$ and se that f separates

\begin{align*}
    \frac{1}{f}\frac{d^2f}{d\phi^2} = \lambda \\
    \frac{1}{r} \sin^2(\theta) \frac{d}{d \rho} \left( \rho^2 \frac{d r(\rho)}{d \rho} \right) + \frac{1}{t} \sin(\theta) \frac{d}{d \theta} \left( \sin(\theta) \frac{d t(\theta)}{d \theta} \right) = \lambda
\end{align*}

Now we can separate the second equation by divinding by $\sin^2(\theta)$

\begin{align*}
    \frac{1}{r}\frac{d}{d\rho}\left(\rho^2 \frac{dr(\rho)}{d\rho}\right) + \frac{1}{t} \frac{1}{\sin(\theta)} \frac{\partial}{\partial \theta} \left( \sin(\theta) \frac{\partial t(\theta)}{\partial \theta} \right) = \frac{\lambda}{\sin^2(\theta)}
\end{align*}

\begin{align*}
    \frac{1}{r}\frac{d}{d\rho}\left(\rho^2 \frac{dr(\rho)}{d\rho}\right) = \frac{\lambda}{\sin^2(\theta)} - \frac{1}{t} \frac{1}{\sin(\theta)} \frac{\partial}{\partial \theta} \left( \sin(\theta) \frac{\partial t(\theta)}{\partial \theta} \right) = \mu
\end{align*}

We now have three separetade ODEs
\begin{align*}
    \frac{1}{f}\frac{d^2f}{d\phi^2} = \lambda \\
    \frac{1}{r}\frac{d}{d\rho}\left(\rho^2 \frac{dr(\rho)}{d\rho}\right) = \mu \\
    \frac{1}{t} \sin(\theta) \frac{d}{d \theta} \left( \sin(\theta) \frac{d t(\theta)}{d \theta} \right) = \lambda - \mu \sin^2(\theta)
\end{align*}

The first equation togheter with its periodic boundry conditions fix $\lambda = m^2$ and $f(\phi) = c_1\cos(m\phi) + c_2\sin(m\phi)$. We can now rewrite the remaining two equations as

\begin{align*}
    \frac{d}{d\rho}\left(\rho^2 \frac{dr(\rho)}{d\rho}\right) -r\mu = 0 \\
    \frac{1}{t} \sin(\theta) \frac{d}{d \theta} \left( \sin(\theta) \frac{d t(\theta)}{d \theta} \right) = \frac{m^2}{\sin^2(\theta)} - \mu
\end{align*}

\begin{align*}
    x
\end{align*}
\section*{Problem 2}

\end{document}

