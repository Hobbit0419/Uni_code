\documentclass[a4paper]{article}
\usepackage[utf8]{inputenc}

\usepackage[swedish]{babel}
\usepackage[pdftex]{graphicx}
\usepackage{amsmath}
\usepackage{float}
\usepackage{caption}
\usepackage{subcaption}
\usepackage{xcolor}

\newcommand{\ihat}{\boldsymbol{\hat{\textbf{\i}}}}
\newcommand{\jhat}{\boldsymbol{\hat{\textbf{\j}}}}
\newcommand{\khat}{\boldsymbol{\hat{\textbf{k}}}}
\newcommand{\placeholder}{{\huge\textbf{\textcolor{red}{Remember to put something good here!!!}}}}

\textwidth 155mm \oddsidemargin -0mm
\parskip 5mm
\parindent 0mm

\title{Homework 2 GoA 3}
\author{Anton Lindbro}
\date{\today}

\begin{document}

\maketitle

\section*{Flux}

In order to compute the flux of the vectorfield $\boldsymbol{F}(x,y,z)=(x^4z,x^3y^2,2x^3yz)$ over the box defined by $0\geq x \geq 1, 0 \geq y \geq 2, 0 \geq z \geq 1$ we use the divergens theorem since we have a vectorfield we can easily calculate the divergens of and a surface that encloses a simple volume the resulting tripple integral will be simple. 

We begin by calculating the divergens of the field F

\begin{align}
    \frac{\partial F_x}{\partial x} &= 4x^3z \\
    \frac{\partial F_y}{\partial y} &= 2yx^3 \\
    \frac{\partial F_z}{\partial z} &= 2x^3y \\
    \nabla \cdot \boldsymbol{F} &= 4x^3(y+z)
\end{align}

\begin{equation}
    \iiint_D 4x^3(y+z) dxdydz
\end{equation}

Where D is the box mentioned above. Using Fubinis therorem we can rewrite it as a iterated integral

\begin{align}
    \int_0^1 dz \int_0^2 dy \int_0^1 4x^3(y+z) dx  \\
    \int_0^1 dz \int_0^2 \left [x^4(y+z)\right ]_0^1 dy &= \int_0^1 dz \int_0^2 y+z dy \\
    \int_0^1 \left [ \frac{y^2}{2}+yz \right ]_0^2 dz &= \int_0^1 2+2z dz \\
    \left [ 2z + z^2 \right ]_0^1 &= 3
\end{align}

Since the divergence theorem gives the wrong orientation we flip the sign and get that the flux over the box is -3

\section*{Series}

In order to determine the convergence of the series below prove that the absolute value of the series converges. 

\begin{equation}
    \sum_{n=1}^{\infty} \frac{(-5)^{n+1}}{(2n)^n}
\end{equation}

Using the root test we find that the series is absolute convergent as follows.

\begin{align}
    \lim_{n\to\infty} \sqrt[n]{\frac{5^{n+1}}{(2n)^n}} \\
    \lim_{n\to\infty} \frac{5^{1+\frac{1}{n}}}{(2n)}
\end{align}

Since the exponent of the numerator goes to one when n grows 12 can be written as 

\begin{equation}
    \lim_{n\to\infty} \frac{5}{(2n)} = 0
\end{equation}

So by the root test the series in 11 is convergent and therfore 10 is absolutely convergent.

\end{document}

