\documentclass[a4paper]{article}
\usepackage[utf8]{inputenc}

\usepackage[swedish]{babel}
\usepackage[pdftex]{graphicx}
\usepackage{amsmath}
\usepackage{float}
\usepackage{caption}
\usepackage{subcaption}
\usepackage{xcolor}
\usepackage{amsfonts}

\newcommand{\ihat}{\boldsymbol{\hat{\textbf{\i}}}}
\newcommand{\jhat}{\boldsymbol{\hat{\textbf{\j}}}}
\newcommand{\khat}{\boldsymbol{\hat{\textbf{k}}}}
\newcommand{\placeholder}{{\huge\textbf{\textcolor{red}{Remember to put something good here!!!}}}}

\textwidth 155mm \oddsidemargin -0mm
\parskip 5mm
\parindent 0mm

\title{Inlämmningsuppgift 3 Algebra 1}
\author{Anton Lindbro}
\date{\today}

\begin{document}

\maketitle

\section{}
Avgör om nedan funktioner är injektiva, surjektiva eller bijektiva

\subsection*{a}
\begin{equation}
    f: \mathbb{Z} \rightarrow \mathbb{Q}, f(x) = \frac{x^3 + 5}{3}
\end{equation}

Vi börjar med att kolla om den är injektiva

\begin{align}
    \frac{x_1^3 + 5}{3} = \frac{x_2^3 + 5}{3}\\
    x_1^3 = x_2^3 \Rightarrow x_1 = x_2 
\end{align}

Så funktionen är injektiv. Funktionen är dock inte surjektiv då alla element i målmängden inte träffas. Funktionen ger ett rationellt tal med 3 i nämnaren och träffar därmed inte alla rationella tal. 

Svar: Funktionen är injektiv

\subsection*{b}
\begin{equation}
    f: \mathbb{R} \rightarrow \mathbb{R}, f(x) = 3x+ \cos{x}
\end{equation}

Även här börjar vi med att kolla om funktionen är injektiva

\begin{align}
    3x_1 + \cos{x_1} = 3x_2 + \cos{x_2} \\
    3(x_1 - x_2) = \cos{x_1} - \cos{x_2}
\end{align}

Eftersom högerledet är begränsad mellan -1 och 1 och vänsterledet är av arbiträrt värde när $x_1 \neq x_2$ så gäller likheten endast när $x_1 = x_2$ alltså är funktionen injektiv. 

Denna funktionen är kontinuerlig och eftersom den linjära termen $3x$ dominerar träffar bilden alla reela tal alltså är den surjektiv.

Eftersom funktionen är både injektiv och surjektiv så är den bijektiv. 

Svar: Funktionen är bijektiv

\end{document}

