\documentclass[a4paper]{article}
\usepackage[utf8]{inputenc}

\usepackage[swedish]{babel}
\usepackage[pdftex]{graphicx}
\usepackage{amsmath}
\usepackage{float}
\usepackage{caption}
\usepackage{subcaption}
\usepackage{xcolor}

\newcommand{\ihat}{\boldsymbol{\hat{\textbf{\i}}}}
\newcommand{\jhat}{\boldsymbol{\hat{\textbf{\j}}}}
\newcommand{\khat}{\boldsymbol{\hat{\textbf{k}}}}
\newcommand{\placeholder}{{\huge\textbf{\textcolor{red}{Remember to put something good here!!!}}}}

\textwidth 155mm \oddsidemargin -0mm
\parskip 5mm
\parindent 0mm

\title{Inlämmingsuppgift 1 Algebra 1}
\author{Anton Lindbro}
\date{\today}

\begin{document}

\maketitle

Visa att om A och B är mängder gäller $\mathcal{P}(A) \cup \mathcal{P}(B) \subseteq \mathcal{P}(A \cup B)$

Eftersom $ \emptyset \subseteq \mathcal{P}(A)$ alltid är sant är påståendet ovan sant.

Alternativt kan man bevisa det

\begin{align}
    \mathcal{P}(A) =& \left\{x | x \subseteq A\right\}\\
    \mathcal{P}(A) \cup \mathcal{P}(B) =& \left\{x | x \subseteq A \lor x \subseteq B\right\}\\
    \left\{x | x \subseteq A \lor x \subseteq B \right\} =& \left\{x | x \subseteq A \cup B\right\} \\
    \left\{x | x \subseteq A \cup B\right\} =& \mathcal{P}(A \cup B)
\end{align}

Alltså
\begin{equation}
    \mathcal{P}(A) \cup \mathcal{P}(B) \subseteq \mathcal{P}(A \cup B)
\end{equation}
\end{document}
