\documentclass[a4paper]{article}
\usepackage[utf8]{inputenc}

\usepackage[swedish]{babel}
\usepackage[pdftex]{graphicx}
\usepackage{amsmath}
\usepackage{float}
\usepackage{caption}
\usepackage{subcaption}
\usepackage{xcolor}

\newcommand{\ihat}{\boldsymbol{\hat{\textbf{\i}}}}
\newcommand{\jhat}{\boldsymbol{\hat{\textbf{\j}}}}
\newcommand{\khat}{\boldsymbol{\hat{\textbf{k}}}}
\newcommand{\placeholder}{{\huge\textbf{\textcolor{red}{Remember to put something good here!!!}}}}

\textwidth 155mm \oddsidemargin -0mm
\parskip5mm
\parindent0mm

\title{Matematisk statestik inlupp 2}
\author{Anton Lindbro}
\date{\today}

\begin{document}

\maketitle

\section{Uppgift 1}

\subsection{(a)}
Täthetsfunktionen för den givna variabeln X fås genom att göra inversmetoden baklänges.

Alltså inverteras den givna funktionen och sedan deriveras den och då erhålls täthetsfunktionen.

\begin{align}
    \frac{1}{3}(1 - \frac{x}{6})
\end{align}

\subsection{(b)}
\begin{figure}[H]
    \begin{small}
        \begin{center}
            \includegraphics[width=0.95\textwidth]{fig_1b.png}
        \end{center}
        \caption{Simulerad fördelning för den givna slumpvariabeln X}
        \label{fig:1b}
    \end{small}
\end{figure}

Som vi ser i figur \ref{fig:1b} så stämmer de simulerade värden ganska väl överens med täthetsfunktionen. 

\section{Uppgift 2}

\begin{figure}[H]
    \begin{small}
        \begin{center}
            \includegraphics[width=0.95\textwidth]{fig_2a.png}
        \end{center}
        \caption{Illustration av kvadrerade Maxwell-Boltzmann fördelningar med variarande argument alpha}
        \label{fig:2a}
    \end{small}
\end{figure}

Större värde på parametern alpha leder till att toppen på fördelningen förskjuts samt att den blir mer utspridd. 

\section{Uppgift 3}

\begin{figure}[H]
    \begin{small}
        \begin{center}
            \includegraphics[width=0.95\textwidth]{fig_3.png}
        \end{center}
        \caption{Illustration av den givna datan med en anpassad linje}
        \label{fig:3}
    \end{small}
\end{figure}

Eftersom kraften är konstant och vi ändrar på massan bör massan ses som den obereonde och accelerationen bör ses som den beroende variabeln. Med det i åtanke anpassades en linje till den givna datan som går att se i figur \ref{fig:3}. Denna linje har ekvationen $-0,143x + 1,794$. Den har en förklaringsgrad på $r^2 = 0.703$. Relativt högt alltså är datan ganska väl förklarad av linjen, det verkar alltså som att förhållandet mellan accelerationen och massan är linjärt. Ett $95\%$ konfidensintervall för kraften ser ut som följande $4,10 < F < 4,82$

\newpage
\section{Uppgift 4}

Mellanankomsttiderna beskrivs av följande statistisk data.

\begin{table}[H]
    \centering
    \caption{}
    \label{tab:my-table}
    \begin{tabular}{ll}
    Medelvärde            & $3,17\cdot 10^{-2}$ \\
    Standardavikelse      & $4,70\cdot 10^{-2}$ \\
    Variationskoefficient & 1,48       \\
    Min                   & 0,00       \\
    Undre kvartilen       & $8,32\cdot 10^{-3}$ \\
    Övre kvartilen        & $4,20\cdot 10^{-2}$ \\
    Max                   & 2,22                     
    \end{tabular}
\end{table}

Fördelningen av mellan ankomsttiderna plottades i ett histogram. 

\begin{figure}[H]
    \begin{small}
        \begin{center}
            \includegraphics[width=0.95\textwidth]{fig_411.png}
        \end{center}
        \caption{Histogram över mellanankomsttider för neutrinos i ICECUBE detektorn}
        \label{fig:tids fördelning}
    \end{small}
\end{figure}

Det syns ganska tydligt att denna verkar följa en exponential fördelning, vilket understyks ytterliggare av att rita in täthetsfunktionen för en exponential fördelning med parameter härled genom moment metoden. 

\begin{figure}[H]
    \begin{small}
        \begin{center}
            \includegraphics[width=0.95\textwidth]{fig_42.png}
        \end{center}
        \caption{Histogram med föreslagen fördelning inritad i orangt}
        \label{fig:fördelning sim}
    \end{small}
\end{figure}

Som jag ser det kan jag inte hitta något som talar emot att denna datan skulle vara exponential fördelad. Momentmetoden gav en intensitet på 31,5

\end{document}