\documentclass[a4paper]{article}
\usepackage[utf8]{inputenc}

\usepackage[swedish]{babel}
\usepackage[pdftex]{graphicx}
\usepackage{amsmath}
\usepackage{float}
\usepackage{caption}
\usepackage{subcaption}
\usepackage{xcolor}

\newcommand{\ihat}{\boldsymbol{\hat{\textbf{\i}}}}
\newcommand{\jhat}{\boldsymbol{\hat{\textbf{\j}}}}
\newcommand{\khat}{\boldsymbol{\hat{\textbf{k}}}}
\newcommand{\placeholder}{{\huge\textbf{\textcolor{red}{Remember to put something good here!!!}}}}

\textwidth 155mm \oddsidemargin -0mm
\parskip 5mm
\parindent 0mm

\title{Inlämmningsuppgift 1}
\author{Anton Lindbro}
\date{\today}

\begin{document}

\maketitle

\section{Problem uppställning}

\begin{figure}[H]
    \begin{small}
        \begin{center}
            \includegraphics[width=0.95\textwidth]{uppställning.jpg}
        \end{center}
        \caption{Figur med krafter och den stående vågen inritad}
        \label{fig:}
    \end{small}
\end{figure}

\begin{align}
    r &= 0,11 \cdot 10^{-3} m\\
    F &= 5,10 \cdot 10^{-3} \cdot 9,82 N\\
    \lambda &= \frac{2 \cdot 0,87}{3} m\\\
    f &= 14,2 Hz
\end{align}

Uppgiften söker kedjans material.

\section{Fysik}

Stående vågor i strängar har några egenskaper som gör dom lätta att hanskas med. Deras våglängd är väl bestämd utifrån strängens längd.

\begin{equation}
    \lambda_n = \frac{2l}{n+1}
\end{equation}

Där n är antalet noder. Våg farten i en sträng är också väl bestämd enligt

\begin{equation}
    v = \sqrt{\frac{F}{\mu}}
\end{equation}

Där F är förspänningen och $\mu$ är den linjära massdensiteten. Våg farten kan också relateras till frekvens och våglängd enligt 

\begin{equation}
    v = \lambda f
\end{equation}

\section{Plan}
För att ta reda på kedjans material behöver vi hitta en densitet. Detta kan göras genom att sätta ekv.2 och 3 lika med varandra och lösa ut $\mu$ och sedan dividera med tvärsnitts arean. När vi sedan har en densitet letar vi upp det material vars densitet matchar närmst i physics handbook. 
\section{Utförande}

\begin{equation}
    \sqrt{\frac{F}{\mu}} = \lambda f \implies \mu = \frac{F}{\lambda^2 f^2}
\end{equation}

Dividerar vi med tvärsnitts arean får vi

\begin{equation}
    \rho = \frac{F}{\lambda^2 f^2 \pi r^2}
\end{equation}

med dom värden givna i uppgiften

Sätter vi in allt detta och beräknar det numeriska värdet får vi $\rho = 19,4 \cdot 10^3$ kg m$^{-3}$. Detta matchar mycket nära densiteten för guld på $19,28 \cdot 10^3$ kg m$^{-3}$

\section{Rimlighet}

För att undersöka svarets rimlighet så gör vi en enhets analys

\begin{align}
    [\rho] = [\frac{F}{\lambda^2 f^2 \pi r^2}]
    kg m^{-3} = kg m s^{-2} m^{-2} s^2 m^{-2}
    kg m^{-3} = kg m^{-3}
\end{align}

Så enheterna stämmer och värdet verkar rimligt då vi kunde hitta ett material som var mycket nära.

\end{document}

