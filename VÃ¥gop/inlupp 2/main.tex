\documentclass[a4paper]{article}
\usepackage[utf8]{inputenc}

\usepackage[swedish]{babel}
\usepackage[pdftex]{graphicx}
\usepackage{amsmath}
\usepackage{float}
\usepackage{caption}
\usepackage{subcaption}
\usepackage{xcolor}

\newcommand{\ihat}{\boldsymbol{\hat{\textbf{\i}}}}
\newcommand{\jhat}{\boldsymbol{\hat{\textbf{\j}}}}
\newcommand{\khat}{\boldsymbol{\hat{\textbf{k}}}}
\newcommand{\placeholder}{{\huge\textbf{\textcolor{red}{Remember to put something good here!!!}}}}

\textwidth 155mm \oddsidemargin -0mm
\parskip 5mm
\parindent 0mm

\title{Vågors fysik med optik inlämmning 2}
\author{Anton Lindbro}
\date{\today}

\begin{document}

\maketitle

\section{Problem uppställning}

Vi har en sond som med hjälp av ett solsegel ska lämna solsystemet. Det finns två möjliga alternativ för material på solseglet och frågan är vilket som bör användas.

\section{Fysiken bakom}

Det som får sonden att drivas framåt med solseglet är stråltrycket från solen. Stråltrycket fås av för absorbenta material

\begin{equation}
    P = \frac{I}{c}
\end{equation}

för reflektiva material

\begin{equation}
    P = \frac{2I}{c}
\end{equation}

Där I är irradiancen som träffar seglet. Om vi antar att solen strålar isotropt fås den av

\begin{equation}
    I = \frac{p_s}{4\pi r^2}
\end{equation}

Gravitationen från solen får vi enligt

\begin{equation}
    F_g = G \frac{mM}{r^2}
\end{equation}

Där m är sondens massa och M är solens massa

Kraften från seglet får av 

\begin{equation}
    F_s = P\cdot A
\end{equation}

Där A är seglets area

\section{Plan}

Sätt $F_s = F_g$ och lös ut A för de två olika materialen och se vilken som ger lägst kostnad

\section{Utförande}

\begin{equation}
    A\frac{P_s}{4\pi r^2 c} = G \frac{mM}{r^2} \Rightarrow A = \frac{4\pi m M c G}{P_s}
\end{equation}

Detta för det absobenta materialet. För det reflektiva materialet blir det bara en faktor $\frac{1}{2}$ skillnad.

Sätter vi in värden får vi 

\begin{align}
    A_r =  A = \frac{4\pi m M c G}{2 P_s} = 3,2 \cdot 10^6\\
    A_a =  A = \frac{4\pi m M c G}{P_s} = 6,4 \cdot 10^6
\end{align}

Kostnaden för det hela kommer vara för den absorbenta vara $6,4x$ kr där x är en arbiträr kvadratmeter kostnad. För den reflektiva kommer det vara $3,2 \cdot 1,6x = 5,12x$ vilket blir billigare. Det reflektiva materialet är det mest kostnads effektiva. 

\section{Rimlighet}
Enhets analys ger

\begin{equation}
    m^2 = kg \cdot kg \cdot m \cdot s^{-1} \cdot m^3 \cdot kg^{-1} \cdot s^{-2} \cdot kg^{-1} \cdot m^{-2} \cdot s^3 
\end{equation}

Den identiteten håller. Den enorma arean är också rimlig med tanke på den lilla kraften som stråltrycket åstadkommer. 

\end{document}

